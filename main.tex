\documentclass[11pt]{amsart}
\usepackage[utf8]{inputenc}
\usepackage{amssymb,amsmath,amsthm}
\usepackage{enumerate}
\usepackage[margin=1in]{geometry}
\usepackage{lineno}
\linenumbers
\usepackage{xcolor}
\usepackage{hyperref}
\hypersetup{
    colorlinks=true,
    linkcolor=blue,
    filecolor=magenta,      
    urlcolor=blue,
    pdftitle={.},
    pdfpagemode=FullScreen,
    }

\urlstyle{same}

\theoremstyle{plain}
\newtheorem{theorem}{Theorem}[section]
\newtheorem{corollary}[theorem]{Corollary}
\newtheorem{lemma}[theorem]{Lemma}
\newtheorem{fact}[theorem]{Fact}
\newtheorem{proposition}[theorem]{Proposition}

\theoremstyle{definition}
\newtheorem{definition}[theorem]{Definition}
\newtheorem{remark}[theorem]{Remark}
\newtheorem{question}[theorem]{Question}
\newtheorem{conjecture}[theorem]{Conjecture}
\newtheorem{convention}[theorem]{Convention}
\newtheorem{hypothesis}[theorem]{Hypothesis}
\newtheorem{example}[theorem]{Example}


\newtheoremstyle{expl}%
{0ex}% Space above
{0ex}% Space below
{}% Body font
{}% Indent amount
{\bfseries}% Theorem head font
{.}% Punctuation after theorem head
{.5em}% Space after theorem head
{}% Theorem head spec (can be left empty, meaning `normal')
\theoremstyle{expl}

\begin{document}
	\title{Comprehensive Exam}
	\date{}
	\author{\small{Shreya Sharma}}
	\maketitle
	\tableofcontents
	
\section{Personal Background}
Below is a list of the mathematics courses which I have taken as a graduate student at the University of South Carolina. The courses that I am currently taking this semester are marked by an asterisk. 
\begin{itemize}
    \item Math 702: Algebra II
    \item Math 704: Analysis II
    \item Math 746: Commutative Algebra 
    \item Math 788: Modular Forms
    \item Math 738: Derived Categories in Algebraic Geometry I
    \item Math 756: Functional Analysis I
    \item Math 780: Elementary Number Theory
    \item Math 750: Fourier Analysis
    \item Math 748: Derived Categories in Algebraic Geometry II
    \item Math 774: Discrete Math I*
    \item Math 748: Homological Algebra* 
\end{itemize}
In summer 2022, I also participated in the seminar on Linear Algebraic Groups and this semester, I am taking part in a student run seminar on Geometric Invariant Theory.
\section{Introduction}
\subsection{Definition and Examples.} Throughout we work over the field of complex numbers $\mathbb{C}$.
A smooth projective variety $X$ is called a \textit{Fano Variety} if its anticanonical divisor $-K_X$ is ample.
For any Cartier divisor $D$ on a variety $X$, $\mathcal{O}_X(D)$ will denote the corresponding invertible sheaf, and, in particular, $\mathcal{O}_X(-K_X)$ is the anticanonical sheaf on $X$ where $-K_X$ is the anticanonical divisor of $X$. 

	To give an example, for any positive integer $n$,  $\mathbb{P}^n$ is a $n$-dimensional Fano variety since its anticanonical sheaf is $\mathcal{O}_{\mathbb{P}^n}(n+1)$ which is ample by II.7 in \cite{Hartshorne}. In fact $\mathbb{P}^1$ is the only 1-dimensional Fano variety up to isomorphism. Fano varieties of dimension $2$ are called \textit{del Pezzo surfaces} and their classification is given in \cite{Manin}, IV \S 24. Clearly, $\mathbb{P}^3$ is a $3$-dimensional Fano variety. To see another example, consider a smooth hypersurface $V$ of degree $d$ in $\mathbb{P}^n$ where $n \geq 2$ and $d<n+1$. Then by the adjunction formula, \cite{Hartshorne}, II.8
	\[
	K_V = (d-n-1)H
	\]
    where $H$ is the hyperplane class of Pic $\mathbb{P}^n$. Then $-K_V$ is ample and so $V$ is a Fano variety.
    
 There are 105 deformation classes of Fano varieties of dimension $3$(or Fano $3$-folds). Fano $3$-folds with Picard rank $\rho =1$ are called \textit{prime} Fano $3$-folds and their classification was first completed by Iskovskikh using the birational method of double projection from a line in \cite{Isk77} and \cite{Isk78}. The classification was later reworked by S. Mukai in \cite{Muk89} using the biregular vector bundle method. So it turns out that there are 17 deformation classes of prime Fano $3$-folds. Fano $3$-folds with $\rho\geq 2$ were all classified by Mori and Mukai in \cite{MM81} and \cite{MM03}. More recently, De Biase, Fatighenti, and Tanturri have obtained a similar description for non-prime Fano $3$-folds by embedding them into products of Grassmannians, \cite{Hove}.
 
My research intends to investigate group of automorphisms of Fano $3$-folds. In this report, I give an outline for the complete classification of Fano $3$-folds given by Iskovskikh, Mori, and Mukai(\S\S 3,4). In \S 5, we discuss some existing results on their automorphism groups. 

\subsection{Notations} In this section, we establish the basic notations and results that will be important for the subsequent classification. 
Let $X$ be a Fano $3$-fold. 
Let $D$ be a divisor on $X$ and let us write $\mathcal{L}$ for the invertible sheaf corresponding to $D$, that is, $\mathcal{L} = \mathcal{O}_X(D)$ and where appropriate, we will write $\omega_X$ for the invertible sheaf associated to the canonical divisor $K_X$.

We also write $H^0(X,D)$ for the finite-dimensional vector space over $\mathbb{C}$ of global sections of $X$.
The symbol $|\mathcal{L}|$ or $|D|$ will denote the complete linear system of effective divisors formed by the divisors of zeroes of sections in $H^0(X,\mathcal{L})$ and we write dim $|D|$ for dim$_{\mathbb{C}}(H^0(X,\mathcal{L})) -1$.
For an arbitrary coherent sheaf $\mathcal{F}$ on $X$, we will write $h^i(X,\mathcal{F})$ for dim $H^i(X,\mathcal{F})$. 

We consider all vector spaces over $\mathbb{C}$. For a $n$-dimensional vector space $V$, Gr$(s,n)$ denotes the space of $s$-dimensional subspaces of $V$, called the Grassmannian. The Picard group of $X$, Pic $X$ is the group of isomorphism classes of invertible sheaves on $X$ under the operation $\otimes$. 

Let $H$ be a divisor on $X$. By $H^3$, we mean the self-intersection number of $H$. See \cite[1.2]{3264} for the definition of intersection number. 
By an anticanonical embedding we will mean the embedding induced by the linear system $|-K_X|$ into a projective space.
\medbreak 
Given a Fano $3$-fold $X$, we label it by a pair of numbers $\rho$-$N$ where $\rho$ is the Picard rank of $X$ and $N$ is the number in the classification found in \cite{FanoV}. A most recent classification table for Fano $3$-folds along with some of their associated invariants and information about their birational geometry, zero section description, etc. can be found on \cite{Fano}.

\subsection{Plan} We begin our study of classification by defining and giving properties of invariants associated to a Fano $3$-fold $X$, viz., Picard rank, index, genus, and degree. The \textit{genus} $g$ of $X$ is the positive integer $(-K_X)^3/2 +1$. The Picard rank $\rho$ is discussed next, it is defined as the rank of the Picard group Pic $X$. The \textit{index} of $X$ is the greatest integer $r$ such that $-K_X = rH$ for some divisor $H$, called the fundamental divisor of $X$. It turns out that the index $r$ satisfies $1 \leq r \leq 4$, we show this in Proposition 2.8. The \textit{degree} $d$ of $X$ is defined as the self-intersection number of $H$. In the case when $-K_X$ is very ample, we give an alternative description for $g$ and $d$ in Proposition 2.4 and Definition 2.9. 

The classification of Fano $3$-folds with Picard rank $1$ is described in section $3$, owing to Iskovskikh and Mukai. In Theorem 3.1, we describe Fano $3$-folds with index $r=3$ and $4$. From Theorem 3.2, we see that Fano threefolds with index $r=2$ have degree $1\leq d \leq 7$, we give a description of them in Theorem 3.3 for each $d$. Through the works of Iskovskikh(\cite{Isk77}, \cite{Isk78}), we have that all Fano $3$-folds with index $r=1$ appear as hyperelliptic varieties, complete intersections or have genus $6 \leq g \leq 12$, $g\neq 11$ except one. We first give hyperelliptic Fano $3$-folds in Theorem 3.5. In Proposition $3.8$, we see that a Fano $3$-fold is a complete intersection in projective space exactly for $g=3,4$, and $5$. A Fano $3$-fold with $g=6$ is given as complete intersection in a Grassmannian, in example 3.8. Finally, Theorem 3.9 establishes the boundedness of the genus $g$ and describes the remaining Fano $3$-folds with $g\geq 6$. 

While Iskovskikh's method gives the existence of prime Fano $3$-folds with genus $g \geq 6$, Mukai gave a more explicit description of such projective varieties as sections of homogeneous spaces, which we describe in Example 3.10. His original result is for more general prime Fano varieties of dimension $n\geq 3$ and index $n-2$, see \cite{Muk89}. Here we state it for $n=3$ in Theorem 3.11. Mukai's method is a vast generalization of Gushel's vector bundle method which he used to describe Fano $3$-folds with genus $6$ and $8$, see \cite{Gus6}, \cite{Gus83}, and \cite{Gus92}. we state Gushel's results in Theorems 3.12 and 3.13.

In section 4, we describe Fano $3$-folds with Picard rank at least $2$ following mainly \cite{MM81} and \cite{MM83}. By a result of Mori and Mukai, the Picard rank of Fano $3$-folds is bounded above(Theorem 4.5). We will call a Fano $3$-fold primitive if it is not isomorphic to the blowing-up of another Fano $3$-fold along a smooth irreducible curve. 
Mori's theory for extremal rays applied to primitive Fano $3$-folds shows that they have $\rho =2 $ or $3$ and they appear as conic bundles over either $\mathbb{P}^2$ or $\mathbb{P}^1 \times \mathbb{P}^1$, see Theorem 4.4. On the other hand, the Fano $3$-folds which are not primitive(called imprimitive) have also been classified into remaining 75 deformation classes. 

\subsection{Basic Properties} We begin by stating some results for smooth projective varieties that will be necessary to arrive at the the definitions of invariants and their properties. 
\begin{proposition}[Adjunction Formula, \cite{Hartshorne}, II.8]
    Let $Y$ be a smooth subvariety of codimension $1$ in a smooth variety $X$ over $\mathbb{C}$. Consider $Y$ as a divisor, and let $\mathcal{L}$ be the associated invertible sheaf on $X$. Then $\omega_Y \cong \omega_X \otimes \mathcal{L}\otimes \mathcal{O}_Y$. In terms of canonical divisors, we have 
    \[
    K_Y = (K_X+Y)|_Y.
    \]
\end{proposition}

The following two theorems will be of fundamental importance in this subsection. We will use these for a Fano $3$-fold $X$. For an alternative formulation to Theorem \ref{C}, see \cite{Laz}, 3.1.
\begin{theorem}[The Kodaira Vanishing Theorem, \cite{Hartshorne}, III] If $X$ is a projective nonsingular variety of dimension $n$ over $\mathbb{C}$ and if $\mathcal{L}$ is an ample invertible sheaf on $X$, then: 
\begin{enumerate}
    \item[a)] $H^i(X, \mathcal{L}\otimes \omega_X) =0$ for $i>0$;
    \item[b)] $H^i(X,\mathcal{L}^{-1})=0$ for $i<n$.
\end{enumerate}
\end{theorem}
\begin{theorem}[Lefschetz Hyperplane Theorem, \cite{Laz}, 3.1]
\label{C}
    Let $X$ be a smooth complex projective variety of dimension $n$, and let $D$ be any effective ample divisor on $X$. Then the restriction 
    \[
    H^i(X,\mathbb{Z}) \rightarrow H^i(D,\mathbb{Z})
    \]
    is an isomorphism for $i\leq n-2$ and injective when $i=n-1$.
\end{theorem}

For a Fano $3$-fold $X$, we define the integer $g=g(X)= -K_X^3/2+1 $ to be the \textit{genus} of $X$.
\begin{proposition}
\label{A}
    If $F \in |-K_X|$ is a smooth surface, $C \in |\mathcal{O}_F(-K_X)|$ is a curve, and $C$ has genus $g=g(C) = h^1(\mathcal{O}_C)$, then the following assertions are true:
    \begin{enumerate}
        \item[(i)] $-K_{X}^3 = 2g-2 $.
        \item[(ii)] If $-K_X$ is very ample, then $\phi_{|-K_X|}(X)= X_{2g-2}$ is a smooth variety of degree $-K_{X}^3 = 2g-2$ in $\mathbb{P}^{g+1}$, the hyperplane sections of which are $K3$ surfaces, and the curves sections of which are canonical curves $C_{2g-2} \subset \mathbb{P}^{g-1}$ of genus $g$.
    \end{enumerate}
\end{proposition}

A sort of converse to Proposition \ref{A} is given in \cite{Isk78}, 1.2.
Note that since $h^1(\mathcal{O}_X)=0$ by the Kodaira Vanishing theorem, the first chern map $H^1(X,\mathcal{O}_X^*) \to H^2(X,\mathbb{Z})$ is injective. Thus the Picard group Pic $X$ coincides with the N\'{e}ron-Severi group $NS(X)$ making Pic $X$ a finitely generated abelian group. The \textit{Picard rank} of $X$ is defined as the rank of Pic $X$, denoted by $\rho(X)$ or simply $\rho$. 
\medbreak
For a Fano $3$-fold $X$, there exists a maximal integer $r = r(X) >0$ such that $-K_X = rH$ for some $H \in \text{Pic}(X)$. We call $r$ the index of $X$ and the divisor $H$ is called a fundamental divisor on $X$.
Iskovskikh's classification relies on the following result.
\begin{theorem}[Shukorov, \cite{Sokudivisor}]
\label{B}
Let $X$ be a Fano $3$-fold. Then the linear system $|H|$ contains a smooth surface.
\end{theorem}
By the Riemann-Roch theorem, Serre duality, the Kodaira vanishing theorem and the adjunction formula we have from \cite{Isk77}, 1.9,
\begin{proposition}
    If $r \geq 2$, then the canonical invertible sheaf of $H$ is given by 
    \[
    \mathcal{O}_H(K_{H}) \simeq \mathcal{O}_H \otimes \mathcal{O}_X(-(r-1)H).
    \]
\end{proposition}
\begin{corollary}
 Let $S \in |H|$ be a smooth surface. Then $S$ is a del Pezzo surface. 	
\end{corollary}
\begin{proof}
 From the proposition and $r\geq 2$, $-K_S = (r-1)H$ is ample. 
\end{proof}
The following result establishes the boundedness of the index of Fano $3$-folds. 
\begin{proposition}
    Let $X$ have index $r\geq 2$, and let $S \in |H|$ be a smooth surface. Then 
    \begin{enumerate}
        \item[(i)] $r \leq 4$;
        \item[(ii)] if $r=2$ then $1 \leq S^3 \leq 9$;
        \item[(ii)] if $r=3$ then $S^3=2$;
        \item[(iv)] if $r=4$ then $S^3=1$.
    \end{enumerate}
\end{proposition}
\begin{proof}
	Let $S \in |H|$ be a del Pezzo surface. Then by \cite{Manin}, IV 
   \[ 1\leq K_S^2 \leq 9.\]
Plugging in the formula for $K_S$ from 2.3, we get 
\[
1 \leq (r-1)^2 S^3 \geq 9.
\]   
Now $S^3$ is a positive integer	as $-K_X$ is ample and $r\geq 2$, so considering possibilities for positive integer values of $S^3$ gives $r\leq 4$. This proves $(i)$. For $(ii)$ and $(iv)$, using $r=2$ and $4$ respectively gives us possible values of $S^3$. If $r=3$, then from the last inequality $S^3 =1$ or $2$. For $S^3=1$, we get a contradiction
\[
2g-2 = -K_X^3 = (3H)^3 = 27,
\]
so $S^3=2$.		
\end{proof}

\begin{definition}
    Set $d = d(X) =H^3$. We call $d$ the \textit{degree} of the Fano $3$-fold $X$. If $H$ is very ample and $r=1$, then $d$ is the same as \textit{degree} of $\phi_{|-K_X|}(X)$ in $\mathbb{P}^{\text{dim}|-K_X|}$.
\end{definition}
\begin{proposition}
\label{pic}
	\begin{enumerate}
		\item[(i)] Pic $X \simeq H^2(X,\mathbb{Z})$.
		\item[(ii)] Pic $X$ is torsion-free. 
	\end{enumerate}
\end{proposition}
\begin{proof}
	The exponential exact sequence 
	\[
	0 \rightarrow \mathbb{Z} \rightarrow \mathcal{O}_X  \xrightarrow{exp} \mathcal{O}_X^* \rightarrow 1
	\]
	 induces the long exact cohomological sequence 
	 \[
	 \dots \rightarrow H^1(X,\mathcal{O}_X) \rightarrow H^1(X,\mathcal{O}_X^*) \xrightarrow{\delta} H^2(X,\mathbb{Z}) \rightarrow H^2(X,\mathcal{O}_X)\rightarrow \dots 
	 \]
	 By the Kodaira vanishing theorem, $H^1(X,\mathcal{O}_X) = H^2(X,\mathcal{O}_X) =0$, so $\delta$ is an isomorphism. This shows $(i)$. Let $S \in |-K_X|$ be a smooth surface. It is a $K3$ surface by \cite{Isk77}, 1.5 and if $r \geq 2$, a del Pezzo surface by Corollary 2.7. In any case, $H^2(S,\mathbb{Z})$ is torsion-free. Further by Lefschetz hyperplane theorem, $H^2(X,\mathbb{Z}) \hookrightarrow H^2(S,\mathbb{Z})$. Thus by $(i)$, Pic $X$ is also torsion-free.
\end{proof}

	  
\section{Fano Threefolds with $\rho =1$}

\subsection{Fano $3$-folds with $r\geq2$}
The following theorems by Iskovskikh(\cite{Isk77}) completely classify Fano $3$-folds with index $r\geq 2$. 
\begin{theorem}
Let $X$ be a Fano $3$-fold of index $r\geq 3$. Then
\begin{enumerate}
\item[(i)] for $r =4$, $\phi_{|H|}: X \xrightarrow{\sim} \mathbb{P}^3$ is an isomorphism,
\item[(ii)] for $r=3$, $\phi_{|H|}: X \xrightarrow{\sim} X_2\subset \mathbb{P}^4$ is an isomorphism of $X$ with a smooth quadric of $\mathbb{P}^4$.
\end{enumerate}
\end{theorem}
The following theorem 3.2 discusses the existence of Fano $3$-folds with index $r=2$ and Theorem 3.3 describes these Fano $3$-folds. With the exceptions of $d=6$ and $7$, each $X$ is a prime Fano $3$-fold.  
\begin{theorem} Let $X$ be a Fano $3$-fold of index $r=2$ and degree $d=H^3$. Then 
        a variety $X$ only exists for $1\leq d \leq 7$, and furthermore
        \begin{enumerate}
        \item[(i)] for $d\geq 3$, $\phi_{|H|}: X \xrightarrow{\sim}X_d \subset \mathbb{P}^{d+1}$ is an embedding of $X$ as a subvariety $X_d$ of degree $d$ in $\mathbb{P}^{d+1}$, with $X_d$ projectively normal.
        \item[(ii)] for $d\geq 4$, $X_d$ is the intersection of the quadrics containing it.
        \end{enumerate} 
        Conversely, for any $d\geq 3$, every smooth projectively normal $3$-fold $X_d \subset \mathbb{P}^{d+1}$ not lying in any hyperplane is a Fano $3$-fold with index $r=2$. For the case when the degree of $\phi_{|H|}$ is $8$, $X_8$ is the image of $\mathbb{P}^3$ in $\mathbb{P}^9$ under the Veronese embedding and so has index $r =4$.
\end{theorem}

\begin{theorem} Let $X$ be as in the previous theorem. Then 
\begin{enumerate}        
        \item[(i)] for $d=7$, $X_7$ is the projection of the Veronese $3$-fold $X_8 \subset \mathbb{P}^9$ from some point of $X_8$.
        \item[(ii)] for $d=6$, $X_6 \simeq \mathbb{P}^1 \times \mathbb{P}^1\times \mathbb{P}^1$ in its Segre embedding.
        \item[(iii)] for $d=5$, $X_5 \subset \mathbb{P}^6$ is unique up to projective equivalence, and can be realized in either of the following equivalent ways:
        \begin{enumerate}
            \item[(a)] as the birational image of a quadric $W \subset \mathbb{P}^4$ under the map defined by the linear system $|\mathcal{O}_W(2)-Y|$ of quadrics passing through a twisted cubic $Y$;
            \item[(b)] as the section of the Grassmannian Gr$(2,5)$ of lines in $\mathbb{P}^4$ by $3$ hyperplanes in general position.
        \end{enumerate}
        \item[(iv)] for $d=4$, $X_4$ is any smooth intersection of two quadrics in $\mathbb{P}^5$.
        \item[(v)] for $d=3$, $X_3$ is any smooth cubic hypersurface of $\mathbb{P}^4$. 
        \item[(vi)] for $d=2$, $X$ can be realized in either of the following equivalent ways:
        \begin{enumerate}
            \item[(a)] as a double cover- the morphism $\phi_{|H|}: X \to \mathbb{P}^3$ is a double covering with smooth ramification surface $D_4 \subset \mathbb{P}^3$ of degree $4$.  
            \item[(b)] as a smooth hypersurface of degree $4$ in $\mathbb{P}(1,1,1,1,2)$.
        \end{enumerate}
        \item[(vii)] for $d=1$, $\phi_{|H|}: X \to \mathbb{P}^2$ is a rational map with a single point of indeterminacy, and with irreducible elliptic fibres; and $X$ can be realized in either of the following equivalent ways:
        \begin{enumerate}
            \item[(a)] $\phi_{|-K_X|}: X \to W_4$ is any double cover of the cone $W_4$ over the Veronese surface $F_4 \subset \mathbb{P}^5$, having smooth ramification divisor $D \subset W_4$ by a cubic hypersurface not passing through the vertex of the cone;
            \item[(b)] any smooth hypersurface of degree $6$ in $\mathbb{P}(1,1,1,2,3)$.
        \end{enumerate}
    \end{enumerate}
\end{theorem}
For the detailed proof of Theorems 3.1-3.3, see $\S\S 4-6$ in \cite{Isk77}. 
\subsection{Fano $3$-folds with $r=1$}
In this subsection, $X$ will denote a Fano $3$-fold of index $r=1$. 

For a Fano $3$-fold, Theorem \ref{B} implies that the fundamental linear system $|H|$ is without fixed components and base points and so by Corollary 2.2 in \cite{Isk77}, it follows that deg $\phi_{|H|} =1$ or $2$. Here we study the case when deg $\phi_{|H|}=2$. 
\begin{definition}
    A Fano $3$-fold $X$ of index $r=1$ is \textit{hyperelliptic} if its anticanonical map $\phi_{|-K_X|}$ is a morphism and is of degree deg $\phi_{|-K_X|} = 2$.
\end{definition}
\begin{theorem}
    Let $X$ be a hyperelliptic Fano variety, and let $\phi_{|-K_X|} : X \to Y \subset \mathbb{P}^{g+1}$ be the corresponding morphism of degree $2$. Then $Y$ is nonsingular and $X$ is uniquely determined by the pair $(Y,D)$, where $D\subset Y$ is the ramification divisor of $\phi_{|-K_X|}$. For $\rho(X)=1$, 
    the pair $(Y,D)$ belongs to one of the following families(and if $D$ is a smooth divisor, then for each pair $(Y,D)$ there exists a Fano $3$-fold $X$):
    \begin{enumerate}
        \item[(i)]$g=2$. $Y \simeq \mathbb{P}^3$, and $D$ is a smooth hypersurface of degree $6$; in this case $X$ can be realized alternatively as a smooth hypersurface of degree $6$ in the weighted projective space $\mathbb{P}(1,1,1,1,3)$. 
        \item[(ii)]$g=3$. $Y \simeq X_2$ is a smooth quadric in $\mathbb{P}^4$ and $D \in |\mathcal{O}_{X_2}(4)|$; that is, $D = X_2 \cap X_4$, where $X_4$ is a quartic in $\mathbb{P}^4$. In this case $X$ can also be realized as a smooth complete intersection in the weighted projective space $\mathbb{P}(1,1,1,1,1,2)$: $X$ is the intersection of a quadric cone and a hypersurface of degree $4$:
        \begin{gather*}
        F_2(x_0,  \dots , x_4)=0 \\
        F_4(x_0, \dots , x_5)=0.
        \end{gather*}
    \end{enumerate}
\end{theorem}
For the proof, see $\S 7, 7.3-7.6$ in \cite{Isk77}.
\medbreak
In \cite{Isk77}, Theorem 3.3 (b), Iskovskikh shows the existence of a Fano $3$-fold of index $1$ whose anticanonical divisor $-K_X$ is not very ample. In Theorem 3.12 (ii), we give a description for it owing to Gushel.
\begin{definition}
    A smooth complete irreducible $3$-fold $X$ over $\mathbb{C}$ is called a \textit{Fano $3$-fold of the principal series} if the anticanonical divisor $-K_X$ is very ample.
\end{definition}
All Fano $3$-folds are of the principal series with the exceptions of hyperelliptic Fano $3$-folds in Theorem 3.5, 1-11(Theorem 3.3, $vii$), and 1-5 b)(Theorem 3.12, $ii$).  From this point, we will consider our Fano $3$-folds to be of principal series and write $X = X_{2g-2} \subset \mathbb{P}^{g+1}$ for a Fano $3$-fold of the principal series in its anticanonical embedding.
For genus $g=3, 4,$ and $5$, and index $1$, we have 
\begin{proposition}
    A Fano $3$-fold $X_{2g-2}\subset \mathbb{P}^{g+1}$ is a complete intersection only for $g=3, 4$ or $5$, and we have that 
    \begin{enumerate}
    \item[(i)]$g=3$. $X_4 \subset \mathbb{P}^4$ is a quartic hypersurface, (1-2 a)).
    \item[(ii)]$g=4$. $X_6 = V_{2\cdot 3}$ is an intersection of a quadric and a cubic in $\mathbb{P}^5$, (1-3)
    \item[(iii)]$g=5$. $X_8 = V_{2\cdot 2\cdot 2}$ is an intersection of $3$ quadrics in $\mathbb{P}^6$, (1-4).
    \end{enumerate}
    Conversely, each smooth complete intersection of the types indicated is a Fano $3$-fold the principal series. 
\end{proposition}
For a proof, see Proposition 1.3 in \cite{Isk78}.
\medbreak
A \textit{line} on $X$ is defined as an effective $1$-dimensional cycle $Z \subset X$ with $-K_X \cdot Z=1$. 
\begin{definition}[Double Projection from a line]
Let $Z$ be a line on $X$ and $H$ denote a hyperplane section of $X$. We define the the double projection from $Z$ as the rational map $\pi_{2Z}: X \to \mathbb{P}^{g-6}$ defined by the linear system $|H-2Z|$ on $X$.
\end{definition}
Shukorov showed the existence of a line on such Fano $3$-folds with Picard rank $1$ in \cite{Sokuline}. Under this assumption and using the method of double projection from a line, Iskovskikh showed that there exist Fano $3$-folds with $\rho =1$ and index $1$ for genus $g \leq 10$ and $g=12$ but not for $g=11$.
 
\begin{example}
        ($g=6$) Let us denote by $X$ a section of Pl\"{u}cker embedding Gr$(2,5) \subset \mathbb{P}^9$ by two hyperplanes in general position and a quadric. By adjunction formula, this is a Fano variety of index $1$, indicated on \cite{Fano} by 1-5 $a)$.
\end{example}
Iskovskikh asserts that it is possible to show that every Fano $3$-fold $X_{10} \subset \mathbb{P}^7$ with Picard rank $1$ is a section of Grassmannian Gr$(2,5)$. See \cite{Isk78}, 1.5.
\medbreak
The following result is the main classification theorem by Iskovskikh for prime Fano $3$-folds with genus $g\geq 7$.
\begin{theorem}
    Let $X = X_{2g-2} \subset \mathbb{P}^{g+1}$ be a Fano $3$-fold with $\rho =1$ and $g\geq 7$. %Suppose that $X$ contains a line
    Let $\pi_{2Z}: X \to W \subset \mathbb{P}^{g-6}$ be the double projection from a sufficiently general line $Z \subset X$. Let $E$ denote the hyperplane section of $W$. Then the following assertions hold:
    \begin{enumerate}
        \item[(i)] $g \leq 12$. 
        \item[(ii)] If $g=12$, then $W=W_5 \subset \mathbb{P}^6$ is a prime Fano $3$-fold with index $2$ and degree $5$ (with possibly one singular point); the map $\rho_Y :W \to X$ inverse to $\pi_{2Z}$ is given by the linear system $|3E-2Y|$, with $Y \subset W$ a normal rational curve of degree $5$ in $\mathbb{P}^5$.
        \item[(iii)] There do not exist any prime Fano $3$-folds with $g=11$. 
        \item[(iv)] If $g=10$, then $W= W_2 \subset \mathbb{P}^4$ is a quadric and $\rho_Y : W \to X$ is given by the linear system $|5E-2Y|$, where $Y$ is a smooth curve of genus $2$ and degree $7$ in $\mathbb{P}^4$.
        \item[(v)] If $g=9$, then $W= \mathbb{P}^3$ and $\rho_Y : \mathbb{P}^3 \to X$ is given by the linear system $|7E-2Y|$, where $Y$ is a smooth curve of genus $3$ and degree $7$.
        \item[(vi)] If $g=8$, then $\pi_{2Z}: X \to \mathbb{P}^2$ is a rational map with fibres(after resolving the determinacy) curves of genus $2$, and such that the inverse images of lines of $\mathbb{P}^2$ are rational surfaces.
        \item[(vii)] If $g=7$, then $\pi_{2Z}: X \to \mathbb{P}^1$ is a rational map whose general fiber(after resolving the indeterminacy) is a del Pezzo surface of degree $5$ with $8$ points blown up; $X$ is a rational $3$-fold, and the projection from a line maps it into a complete intersection of $3$ quadrics of $\mathbb{P}^6$ containing a smooth rational ruled surface $R_3 \subset \mathbb{P}^4$.
        \end{enumerate}
\end{theorem}
The proof of $(i)$ and the construction of Fano $3$-folds with given genus in $(ii)$-$(vii)$ uses the birational technique of projection and double projection from lines on $X$. See \cite{Isk78}, $\S 6$ for details.
For an alternative proof of boundedness of genus for Fano $3$-folds using vector bundles, see \cite{Muk92}.
	    
\subsection{Mukai and Gushel's description of Fano $3$-folds with $g\geq 6$}
To classify Fano $n$-folds($n\geq 3$) of the principal series of genus $g \geq 6$, Mukai first showed the existence of a vector bundle $\mathcal{E}$ on $X$. Then using the linear system $|\mathcal{E}|$, we can embed $X$ into a Grassmannian variety and describe its image. We find that $X$ is a linear section of some homogeneous space. We describe these homogeneous spaces in the following examples.
\begin{example}\begin{enumerate}\item[(i)] $g=7$. Let $V$ be a $9$-dimensional vector space with $F$ as a non degenerate symmetric bilinear form on $V$ and $S$ be the space of spinors of $F$. Denote by $\Sigma_{12}^{10} \subset $Gr$(4,9)$ the set of all $4$-dimensional subspaces $W$ of $V$ with $F(W,W)=0$. Then $\Sigma_{12}^{10}$ is a smooth $10$-dimensional subvariety of Grassmannian Gr$(4,9)$ and can be embedded in $\mathbb{P}^{15}$ by the spinor coordinates. 

Here $\Sigma^{10}_{12} = SO_{10}(\mathbb{C})/P$ is a homogeneous space with $P$ a maximal parabolic subgroup of $SO_{10}(\mathbb{C})$ and is unique up to isomorphism.
\item[(ii)] $g=8$. The Grassmannian $\Sigma^8_{14}:=$ Gr$(2,6) \subset \mathbb{P}(\wedge^2 \mathbb{C}^6) = \mathbb{P}^{14}$ is a smooth $8$-dimensional Fano variety of index $8$.  Here $\Sigma^8_{14} = SL_6(\mathbb{C})/P$ is a homogeneous space with $P$ a maximal parabolic subgroup of $SL_6(\mathbb{C})$.
      \item[(iii)] $g=9$. Let $V$ be a $6$-dimensional vector space and $F$ be a non degenrate skew-symmetric bilinear form on $V$. Let us denote by $\Sigma_{16}^6$ the set of all $3$-dimensional subspaces $W$ of $V$ such that $F(W,W)=0$. Then $\Sigma_{16}^6 \subset $ Gr$(3,6)$ is a homogeneous space isomorphic to $U(3)/O(3)$ as varieties, hence is a smooth $6$-dimensional subvariety of degree $16$ in Gr$(3,6)\subset  \mathbb{P}^{19}$.
            \item[(iv)] $g=10$. Let $V$ be a $7$-dimensional vector space and $F$ be a non degenerate skew-symmetric $4$-linear form on $V$. Denote by $\Sigma_{18}^5$ the set of all $5$-dimensional subspaces $W$ of $V$ such that $F(W,W,W,W)=0$. Then $\Sigma_{18}^5 \subset $ Gr$(5,7)$ is a homogeneous space, hence a smooth $5$-dimensional subvariety of degree $18$ in Gr$(5,7) \subset \mathbb{P}^{20}$. Here $\Sigma^5_{18}$ is a homogeneous space under the action of the exceptional Lie group of type $G_2$ and is isomorphic to $G_2/P$ where $P$ is a maximal parabolic subgroup of $G_2$.
            
            \item[(v)] $g=12$. Let $V$ be a $7$-dimensional vector space and $F_1$, $F_2$, and $F_3$ be three linearly independent skew-symmetric bilinear forms on $V$. Denote by $X$ the set of $3$-dimensional subspaces $W$ of $V$ with $F_1(W,W)=F_2(W,W)=F_3(W,W)=0$. If the subspace $F_1 \wedge V^{\vee} + F_2 \wedge V^{\vee} + F_3 \wedge V^{\vee}$ of $\wedge^3 V^{\vee}$ contains no vectors of the form $f_1 \wedge f_2 \wedge f_3 \neq 0$ for $f_1, f_2, f_3 \in V^{\vee}$ then $X$ is a smooth $3$-dimensional subvariety of degree $22$, denoted by $\Sigma_{22}^3$.
        \end{enumerate}
\end{example}

Let us write $n(g)$ for the dimension of the varieties in the above example for a given genus $g$. The isomorphism class of $\Sigma_{2g-2}^{n(g)}$ depends only on the vector space spanned by a multilinear form $F$ when $g\neq 12$ and by $F_1, F_2,$ and $F_3$ for $g=12$. So we denote $\Sigma_{2g-2}^{n(g)}$ by $\Sigma_g(V,M)$. For $g=8$, we take $M=0$.  
\begin{theorem}
    Let $X$ be a prime Fano $3$-fold of index $1$ and genus $g$, $6\leq g \leq 10$ over $\mathbb{C}$. Then there exists a vector space $V$ and a space $M$ of multilinear forms on $V$ such that $X$ is isomorphic to a linear section of $\Sigma_g(V,M) \subset \mathbb{P}^{g+1}_{\mathbb{C}}$.
    For $g=12$, then $X$ is isomorphic to some threefold from Example 3.11, (v).
\end{theorem}
Mukai showed the theorem for $n$-dimensional($n\geq 3$) Fano manifolds with index $n-2$ and genus $g\geq 6$, see \cite{Muk89}.
In particular, for $n=3$ and $7 \leq g \leq 10$, a Fano $3$-fold $X_{2g-2} \subset \mathbb{P}^{g+1}$ is obtained as a complete intersection of the homogeneous space $\Sigma_{2g-2}^{n(g)}$ and a linear subspace of codimension $n(g)-3$ in $\mathbb{P}(V) = \mathbb{P}^{g+n(g)-2}$. And by Lefschetz theorem, it follows that such $X=X_{2g-2}$ has Pic $(X) \simeq \mathbb{Z}(-K_X)$.
For Fano $3$-folds of genus $6$ and $8$ over $\mathbb{C}$, the theorem was proved independently by Gushel. For $g=6$, we have  
\begin{theorem}[Gushel, \cite{Gus6}]
    Let $X=X_{10} \subset \mathbb{P}^7$ be an anticanonically embedded Fano threefold of index $1$ and genus $6$ with $\rho(X)=1$. Then $X$ is one of the following:
    \begin{enumerate}
        \item[(i)] a section of the Grassmannian $Gr(2,5)$ embedded by Pl\"{u}cker embedding into $\mathbb{P}(\wedge^2 \mathbb{C}^5)$ by a subspace of codimension $2$ and a quadric,
        \item[(ii)] the section by a quadric of a cone $W=W_5 \subset\mathbb{P}^7$ over a nonsingular del Pezzo threefold $V= V_5 \subset \mathbb{P}^6$ of degree $5$.
    \end{enumerate}
    Threefolds of type $(i)$ and $(ii)$ are not isomorphic.
\end{theorem}
See \cite{Gus6} for details on the proof.
A similar method can be applied to study Fano threefolds of genus $8$. 
\begin{theorem}[Gushel, \cite{Gus83}, \cite{Gus92}]
    Let $X=X_{14} \subset \mathbb{P}^9$ be an anticanonically embedded Fano threefold of index $1$ and genus $8$ with $\rho(X)=1$. Then $X$ is a section of the Grassmannian Gr$(2,6)$ embedded by Pl\"{u}cker embedding into $\mathbb{P}(\wedge^2 \mathbb{C}^6) = \mathbb{P}^{14}$ by a subspace of codimension $5$.
\end{theorem}
\medbreak
As a consequence, we see that the Fano $3$-folds with Picard rank $\rho =1$ appear as one of the following: 
\begin{enumerate}
    \item[(i)] Sections of Grassmannians(1-5 a), 1-6, 1-7, 1-8, 1-9, 1-10, 1-15)
    \item[(ii)] Complete intersections in projective or weighted projective space (1-2 a), 1-3, 1-4, 1-11, 1-12, 1-13, 1-14, 1-16, 1-17, 1-11, 1-12)
    \item[(iii)] Hyperelliptic (1-1, 1-2 b))
    \item[(iv)] double cover of a section of a Grassmannian (1-5 b))
\end{enumerate}

\section{Fano Threefolds with $\rho \geq 2$}
We begin by noting that for a Fano $3$-fold $X$, $B_2 =\rho$ where $B_2$ is the second betti number of $X$. We will use $\rho$ instead of $B_2$ as was used in Mori and Mukai's original work. 
In this section, we consider Fano threefolds with $\rho \geq 2$. 
In \cite{MM81} and \cite{MM03}, Mori and Mukai announced the classification result for such Fano threefolds: 
	\begin{theorem}
	There are exactly $88$ types of Fano $3$-folds with $\rho \geq 2$ up to deformations.
	\end{theorem}
In \cite{MM83}, they explained the general principle of how to classify Fano $3$-folds with $\rho \geq 2$ and an outline of the proof for some cases. See also \cite{MM85}, \S 7 for the proof of the following assertions pertinent to the classification:
an arbitrary smooth $3$-fold in each of the $88$ deformation classes is a Fano $3$-fold with $\rho \geq 2$, two arbitrary Fano $3$-folds in different classes are not deformation equivalent to each other.
	We begin with some definitions. 
	\begin{definition}
		A Fano $3$-fold is \textit{imprimitive} if it is isomorphic to the blow-up of a Fano $3$-fold along a smooth irreducible curve. A Fano $3$-fold is \textit{primitive} if it is not imprimitive.
	\end{definition}

	\begin{definition}
	    A smooth variety over a smooth surface $S$ is a conic bundle if every geometric fibre of $X \to S$ is isomorphic to a conic, i.e., a scheme of zeroes of a non zero homogeneous form of degree $2$ on $\mathbb{P}^2$.
	\end{definition}
    It is clear that prime Fano $3$-folds we saw in $\S 3$ are primitive. So in what follows, we only discuss higher Picard rank case. The following theorem is the first step towards the classification.
    \begin{theorem}
    	Let $X$ be a primitive Fano $3$-fold. Then we have 
    	\begin{enumerate}
    		\item $\rho \leq 3$,
		\item if $\rho =2$, then $X$ is a conic bundle over $\mathbb{P}^2$, and there are $9$ deformation types, 
    		\item if $\rho =3$, then $X$ is a conic bundle over $\mathbb{P}^1 \times \mathbb{P}^1$ and has $4$ deformation types. 
    	\end{enumerate}
    \end{theorem}
    The proof is based on Mori's theory of extremal rays for Fano threefolds. See \cite{MM83}, \S8 for details.
    \medbreak
    An imprimitive Fano $3$-fold can be obtained from successive curve-blow-ups of primitive Fano $3$-folds by using their conic bundle structure or the existence of lines on Fano $3$-folds with $\rho=2$. In the latter case, there can be several possibilities for each of the extremal rays and each possibility leads to an imprimitive Fano $3$-fold with $\rho=2$. Combining all this information we get \cite[Table 2]{MM81}.
    
    Since the blowing-up of a Fano $3$-fold along a non-empty smooth irreducible curve $C$ increases $\rho$ by $1$(\cite{MM85}, 2.1), we obtain imprimitive Fano $3$-folds with $\rho \geq 3$. See Propositions 7-9 in \cite{MM81} for the strong necessary conditions that $C$ must satisfy for the blow-up to be a Fano $3$-fold. These successive curve blow-ups result in \cite[Tables 3-5]{MM81}.
    \medbreak
    The following theorem summarizes the classification for imprimitive Fano $3$-folds. 
    
    \begin{theorem}
        Let $X$ be an imprimitive Fano $3$-fold. Then we have
        \begin{enumerate}
            \item[(i)] $2\leq \rho \leq 10$.
            \item[(ii)] if $\rho =2$, there are 27 deformation types.
            \item[(iii)] if $\rho=3$, there are $27$ deformation types.
            \item[(iv)] if $\rho=4$, there are $13$ deformation types.
            \item[(v)] if $\rho=5$, there are $3$ deformation types.
            \item[(vi)] if $\rho \geq 6$, then $X$ is isomorphic to $\mathbb{P}^1 \times S$ where $S$ is a del Pezzo surface of degree $11-\rho$. There is one deformation class for each $\rho$ in this case. 
        \end{enumerate}
    \end{theorem}
    
\section{Automorphism group of Fano threefolds}
    For a Fano variety $X$, a natural question to ask is what the group of its biregular automorphisms Aut$(X)$ looks like. 
    In dimension $1$, the only Fano variety is $\mathbb{P}^1$ and it has automorphism group PGL$(2,\mathbb{C})$ (\cite{Hartshorne}, II.7.1.1). In dimension $2$, the detailed structure of automorphism groups of del Pezzo surfaces is also known, see \cite{Del}. Such a description of automorphism groups is far from being complete in the case of dimension $3$. 
    
     A theorem of Kuznetsov, Prokhorov, and Shramov states that for a prime Fano $3$-fold $X$, Aut$(X)$ is finite unless it is $\mathbb{P}^3$, a smooth quadric $Q$ in $\mathbb{P}^4$, Fano $3$-fold $X_5$, or has genus $12$ in which case it is one of the three threefolds-- $X^{\text{MU}}_{22}$, $X^a_{22}$, $X^{m}_{22}$. See \cite{Hil} for a description of these threefolds and the proof of the theorem. In \cite{Cubic}, the authors have classified groups which faithfully act on smooth cubic threefolds over $\mathbb{C}$.
    
    More recently, automorphism groups of Fano threefolds of higher Picard rank have also been studied. In \cite{Inf} Theorem 1.2, the authors have shown that the automorphism group Aut$(X)$ is finite unless $X$ is contained in some 62  deformation families. The connected component of the identity Aut$^0(X)$ in Aut$(X)$ is also described for all Fano $3$-folds $X$, see the classification table in \cite{Inf} or \cite{Fano}.
    

\begin{thebibliography}{26}
\bibitem{Hil}
A. G. Kuznetsov, Yu. G. Prokhorov and C. A. Shramov,
\newblock{\em Hilbert schemes of lines and conics and automorphism groups of Fano threefolds},
\newblock Japan. J. Math. \textbf{13}, No. 1 (2018), 109-185.

\bibitem{3264}
D. Eisenbud, J. Harris,
\newblock{\em 3264 and All That: A Second Course in Algebraic Geometry}
\newblock Cambridge: Cambridge University Press, (2016).

\bibitem{Hove}
De Biase, L., Fatighenti, E., Tanturri, F.
\newblock{ \em   Fano 3-folds from homogeneous vector bundles over Grassmannians.}
\newblock Rev Mat Complut (2021).

\bibitem{Del}
I. V. Dolgachev and V. A. Iskovskikh, 
\newblock{\em Finite subgroups of the plane Cremona group},
\newblock Algebra. arithmetic, and geometry, vol. I, In honor of Yu. I. Manin, Progr. Math., vol. 269, Birkh\"{a}user Boston, Inc., Boston, MA (2009), 443-548.

\bibitem{Cubic}
Li Wei, Xun Yu,
\newblock{\em Automorphism groups of smooth cubic threefolds},
\newblock J. Math. Soc. Japan \textbf{72}, No. 4 (2020), 1327-1343.

\bibitem{Gus6}
N. P. Gushel,
\newblock{\em Fano varieties of genus $6$},
\newblock Izv. Akad. Nauk. SSSR Ser. Mat. \textbf{46} (1982), 1159-1174 (Russian); English transl. in Math. USSR  Izv. \textbf{21} (1983), 445-459.

\bibitem{Gus83}
N. P. Gushel,
\newblock{\em Fano varieties of genus $8$},
\newblock Usp. Mat. Nauk \textbf{38} (1983), 163-164 (Russian); English transl. in Russ. Math. Surv. \textbf{38} (1983), 192-193. 

\bibitem{Gus92}
N. P. Gushel,
\newblock{\em Fano $3$-folds of genus $8$},
\newblock Algebra i Analiz \textbf{4} (1992), 120-134 (Russian); English transl. in St. Petersbg. Math. J. \textbf{4} (1993), 115-129. 

\bibitem{Fano}
Pieter Belmans,
\newblock{\em \url{https://www.fanography.info/}}

\bibitem{Hartshorne}
R. Hartshorne.
\newblock {\em Algebraic Geometry},
\newblock Graduate Texts in Mathematics, No. 52, Springer-Verlag, New York-Heidelberg, 1977.

\bibitem{Laz}
R. Lazarsfeld,
\newblock{\em Positivity in Algebraic Geometry I},
\newblock Classical Setting: Line Bundles and Line Series. Springer-Verlag, Berlin, (2004). 

\bibitem{MM81}
S. Mori and S. Mukai,
\newblock{\em Classification of Fano $3$-Folds with $B_2 \geq 2$.}
\newblock Manuscripta Math. \textbf{36} (1981), 147-162.

\bibitem{MM03}
S. Mori and S. Mukai,
\newblock{\em Erratum: Classification of Fano $3$-Folds with $B_2 \geq 2$.}
\newblock Manuscripta Math. \textbf{110}, (2003), 407.

\bibitem{MM83}
S. Mori and S. Mukai,
\newblock{\em On Fano $3$-folds with $b_2 \geq 2$}. 
\newblock Advanced Studies in Pure Mathematics 1, "Algebraic Varieties and Analytic Varieties" 101-129, Kinokuniya Co. and North-Holland Publ. Co., Tokyo and Amsterdam, 1983.

\bibitem{MM85}
S. Mori and S. Mukai,
\newblock{\em Classification of Fano $3$-Folds with $b_2 \geq 2$, I}.
\newblock Algebraic and Topological Theories. Papers from the symposium dedicated to the memory of Dr. Takehiko Miyata held in Kinosaki, (1985), 496-545.

\bibitem{Muk89}
S. Mukai,
\newblock{\em Biregular classification of Fano 3-folds and Fano manifolds of coindex 3},
\newblock Proc. Nat. Acad. Sci. U.S.A., \textbf{86}, No. 9 (1989), 3000-3002.

\bibitem{Muk92}
S. Mukai, (1992). 
\newblock{\em Fano 3-folds},
\newblock In G. Ellingsrud, C. Peskine, G. Sacchiero, \& S. Stromme (Eds.), Complex Projective Geometry: Selected Papers (London Mathematical Society Lecture Note Series, pp. 255-263). Cambridge: Cambridge University Press. 

\bibitem{Isk77}
V.A. Iskovskih, 
\newblock {\em Fano 3-folds I},
\newblock Izv. Akad. Nauk SSSR Ser. Mat.  \textbf{41} (1977), 516-562; English transl. in Math. USSR Izv. \textbf{11} No. 3 (1977).

\bibitem{Isk78}
V. A. Iskovskih, 
\newblock {\em Fano 3-folds II}, 
\newblock Izv. Akad. Nauk SSSR Ser. Mat.
\textbf{42} (1978), 506-549; English transl. in Math. USSR Izv. \textbf{12} No. 3 (1978), 469-506.

\bibitem{FanoV}
V. A. Iskovskih, Yu. G. Prokhorov, Fano varieties,
\newblock{ \em Algebraic Geometry V},
\newblock Encyclopaedia Math. Sci., vol. 47, Springer, Berlin 1999.

\bibitem{Inf}
V. V. Przyjalkowski and I. A. Cheltsov and K. A. Shramov,
\newblock{\em Fano threefolds with infinite automorphism groups}, 
\newblock Izvestiya: Mathematics \textbf{83}, No. 4 (2019), 860-907.

\bibitem{Sokudivisor}
V. V. \v{S}okurov, 
\newblock{\em The smoothness of the general anticanonical divisor on a Fano variety},
\newblock Izv. Akad. Nauk SSSR Ser. Mat. \textbf{43} (1979), 430-441; English transl. in Math. USSR Izv. \textbf{14} No. 2 (1980).

\bibitem{Sokuline}
V. V, \v{S}okurov, 
\newblock {\em The Existence of lines on Fano threefolds},
\newblock English transl. in Math. USSR Izv. \textbf{15} (1980), 173-209.

\bibitem{Manin}
Yu. I. Manin,
\newblock{\em Cubic Forms- Algebra, Geometry, Arithmetic}.
\newblock 2nd Edition, (1986). 

\end{thebibliography}
\end{document}

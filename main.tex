\documentclass[12pt]{amsart}
\usepackage[utf8]{inputenc}
\usepackage{amssymb,amsmath,amsthm}
\usepackage[margin=1in]{geometry}
\usepackage{xcolor}

\begin{document}
	\title{\large{Some title goes here...}}
	\date{}
	\author{\small{Shreya Sharma}}
	\maketitle
	\tableofcontents
	
\section{Introduction}
\noindent Throughout we work over the field of complex numbers $\mathbb{C}$. \\
\textcolor{red}{Q.1 Fano varieties/ 3-folds: if former then discuss dim 1,2 situation(necessary?), if latter just discuss 3-folds examples}
A nonsingular projective variety $X$ is called a \textit{Fano Variety} if the anticanonical divisor $-K_X$ is ample. In this paper/article, we aim to understand/see the classification of Fano 3-folds. \\

	The following definitions and properties follow \textcolor{cyan}{Isko-1}. For any Cartier divisor $D$ on a variety $X$, $\mathcal{O}_X(D)$ wll denote the corresponding invertible sheaf, and, in particular, $\mathcal{O}_X(-K_X)$ is the canonical sheaf on $X$ where $-K_X$ is the canonical divisor of $X$. \\
	\textcolor{red}{ Q.2 Complete linear system, $h^i(X)$, Proposition 1.3, build upto Prop 1.6 \& as a consequence genus, and the following para- NS(X) =Pic(X) has no torsion, $\rho=b_2$, fundamental divisor $H$, index. Then Def 1.13 for degree.}\\
	  
	\textcolor{olive}{Notations, definition(index, $\rho$, degree, etc.), basic examples, goal(s), other preliminaries about invariants, definitions, etc.}
	\section{Fano Threefolds with $\rho =1$}
	\textcolor{olive}{ $V_d$ notation, etc that will be used in next section}
	\section{Fano Threefolds with $\rho \geq 2$}
	\textcolor{olive}{Follow goodnotes file: primitive and imprimitive classification separately, basically follow Mori and Mukai}\\ 
	\textcolor{red}{Q. Tables???}
	\section{References}
\end{document}

\documentclass[11pt]{amsart}
\usepackage[utf8]{inputenc}
\usepackage{amssymb,amsmath,amsthm}
\usepackage{enumerate}
\usepackage[margin=1in]{geometry}
\usepackage{xcolor}


\theoremstyle{plain}
\newtheorem{theorem}{Theorem}[section]
\newtheorem{corollary}[theorem]{Corollary}
\newtheorem{lemma}[theorem]{Lemma}
\newtheorem{fact}[theorem]{Fact}
\newtheorem{proposition}[theorem]{Proposition}

\theoremstyle{definition}
\newtheorem{definition}[theorem]{Definition}
\newtheorem{remark}[theorem]{Remark}
\newtheorem{question}[theorem]{Question}
\newtheorem{conjecture}[theorem]{Conjecture}
\newtheorem{convention}[theorem]{Convention}
\newtheorem{hypothesis}[theorem]{Hypothesis}
\newtheorem{example}[theorem]{Example}


\newtheoremstyle{expl}%
{0ex}% Space above
{0ex}% Space below
{}% Body font
{}% Indent amount
{\bfseries}% Theorem head font
{.}% Punctuation after theorem head
{.5em}% Space after theorem head
{}% Theorem head spec (can be left empty, meaning `normal')
\theoremstyle{expl}

\begin{document}
	\title{Comprehensive Exam}
	\date{}
	\author{\small{Shreya Sharma}}
	\maketitle
	\tableofcontents
	
\section{Personal Background}

\section{Introduction}
\subsection{Definition and Examples.} Throughout we work over the field of complex numbers $\mathbb{C}$.
A smooth projective variety $X$ is called a \textit{Fano Variety} if its anticanonical divisor $-K_X$ is ample.
For any Cartier divisor $D$ on a variety $X$, $\mathcal{O}_X(D)$ will denote the corresponding invertible sheaf, and, in particular, $\mathcal{O}_X(-K_X)$ is the anticanonical sheaf on $X$ where $-K_X$ is the anticanonical divisor of $X$. 

	To give an example, for any positive integer $n$,  $\mathbb{P}^n$ is a $n$-dimensional Fano variety since its anticanonical sheaf is $\mathcal{O}_{\mathbb{P}^n}(n+1)$ which is ample by II.7 in \cite{Hartshorne}. In fact $\mathbb{P}^1$ is the only 1-dimensional Fano variety up to isomorphism. Fano varieties of dimension $2$ are called \textit{del Pezzo surfaces} and their classification is given in \textcolor{cyan}{cite}. Clearly, $\mathbb{P}^3$ is a $3$-dimensional Fano variety. To see another example, consider a smooth hypersurface $V$ of degree $d$ in $\mathbb{P}^n$, $n \geq 2$. Then by the adjunction formula, \cite{Hartshorne}, II.8
	\[
	K_V = (d-n-1)H
	\]
    where $H$ is the generator of Pic $\mathbb{P}^n$. Then if $d-n-1 <0$, $-K_V$ is ample and so $V$ is a Fano variety.
    
 Fano varieties of dim $3$ with Picard rank $\rho =1$ are called \textit{prime} Fano $3$-folds and their classification was first completed by Iskovskikh using the birational method of double projection from a line in \cite{Isk77} and \cite{Isk78}. The classification was later reworked by S. Mukai in \cite{Muk89} using the biregular vector bundle method. Fano $3$-folds with $\rho\geq 2$ were all classified by Mori and Mukai in \textcolor{cyan}{to cite}. More recently, De Biase, Fatighenti, and Tanturri have obtained a description of a general member of each deformation family of Fano $3$-folds as the zero locus of a homogeneous vector bundle in a product of Grassmannians and weighted projective spaces, \textcolor{cyan}{cite}.\\
The subject of this paper/article/survey/document is Fano $3$-folds, we study their classical classification given by Iskovskikh, Mukai, and Mori. 

\subsection{Potential Research} 

\subsection{Plan.} We begin by giving a Iskovskih's proof of the boundedness of the index $r$ of smooth Fano $3$-folds in the next subsection. In section 3, we describe he classification for Fano $3$-fold with Picard rank $1$. Iskovskih classified all smooth Fano $3$-folds over $\mathbb{C}$ with index $r\geq 2$ in \cite{Isk77}.  We describe them in 3.1. In 3.2, we describe hyperelliptic Fano $3$-folds following \cite{Isk77}. It turns out all Fano $3$-folds except a few have very ample anticanonical divisor and $r=1$. An outline of their classification is given in 3.3 following \cite{Isk78}.\\
S. Mukai gave a more explicit description of Fano $3$-folds with genus $g\geq 7$ and Gushel did it independently for $g=6,8$. We describe these in 3.4. The exposition here primarily follows \cite{Muk89} and \textcolor{cyan}{cite}.\\
In section 4, we describe Fano $3$-folds with Picard rank at least $2$ following \textcolor{cyan}{Multiple citations}. 

\subsection{Notations}
In this section, we establish the basic results that are important for subsequent classification. 
Let $X$ be a Fano $3$-fold. 
Let $D$ be a divisor(class) on $X$ and let us write $\mathcal{L}$ for the invertible sheaf(class) corresponding to $D$, that is, $\mathcal{L} = \mathcal{O}_X(D)$. We also write $H^0(X,D)$ for the finite-dimensional vector space over $\mathbb{C}$ of global sections of $X$.
The symbol $|\mathcal{L}|$ or $|D|$ will denote the complete linear system of effective divisors formed by the divisors of zeroes of sections in $H^0(X,\mathcal{L})$. Also we write dim $|D|$ for dim$_{\mathbb{C}}(H^0(X,\mathcal{L})) -1$.
For an arbitrary coherent sheaf $\mathcal{F}$ on $X$, we will write $h^i(X,\mathcal{F})$ for dim $H^i(X,\mathcal{F})$. We consider all vector spaces over $\mathbb{C}$. For a $n$-dimensional vector space $V$, Gr$(s,n)$ denotes the space of $s$-dimensional subspaces of $V$, called the Grassmannian.\\

\noindent For a Fano $3$-fold $X$, we define the integer $g=g(X)= -K_X^3/2+1 $ to be the \textit{genus} of $X$.
\begin{proposition}
\label{A}
    If $F \in |-K_X|$ is a smooth surface, $C \in |\mathcal{O}_F(-K_X)|$ is a curve, and $C$ has genus $g=g(C) = h^1(\mathcal{O}_C)$, then the following assertions are true:
    \begin{enumerate}
        \item[(i)] $-K_{X}^3 = 2g-2 $.
        \item[(ii)] If $-K_X$ is very ample, then $\phi_{|-K_X|}(X)= X_{2g-2}$ is a smooth variety of degree $-K_{X}^3 = 2g-2$ in $\mathbb{P}^{g+1}$, the hyperplane sections of which are $K3$ surfaces, and the curves sections of which are canonical curves $C_{2g-2} \subset \mathbb{P}^{g-1}$ of genus $g$.
    \end{enumerate}
\end{proposition}

\noindent A sort of converse to Proposition \ref{A} is given in \cite{Isk78}, 1.2.\\
\noindent Note that since $h^1(\mathcal{O}_X)=0$ by Kodaira Vanishing theorem, the first chern map $H^1(X,\mathcal{O}_X^*) \to H^2(X,\mathbb{Z})$ is injective. Thus the Picard group Pic $X$ coincides with the N\'{e}ron-Severi group $NS(X)$ making Pic $X$ a finitely generated abelian group. The \textit{Picard rank} of $X$ is defined as the rank of Pic $X$, denoted by $\rho(X)$ or simply $\rho$.

\noindent The following result is due to \v{S}okurov \cite{Sokudivisor}.
\begin{theorem}
\label{B}
Let $X$ be a Fano $3$-fold. There exists a divisor $H \in$ Pic$(X)$ and a natural number $r$ such that $-K_X = rH$ and the linear system $|H|$ contains a smooth surface.
\end{theorem}
The maximal such integer $r$ is called the \textit{index} of the Fano $3$-fold $X$. 
By the Riemann-Roch theorem, Serre duality, the Kodaira vanishing theorem and the adjunction formula we have from \cite{Isk77}, 1.9,
\begin{proposition}
    (\textcolor{olive}{fix notation- H should be S})If $r \geq 2$, then the canonical invertible sheaf of $H$ is given by 
    \[
    \mathcal{O}_H(K_{H}) \simeq \mathcal{O}_H \otimes \mathcal{O}_X(-(r-1)H).
    \]
\end{proposition}
\textcolor{olive}{proof does not seem too important, so not writing it. But I am not sure. maybe I should omit thispropsotion here and include it in the proof of the following corollary. Think and Ask!}\\
\begin{corollary}
 Let $S \in |H|$ be a smooth surface. Then $S$ is a del Pezzo surface. 	
\end{corollary}
\begin{proof}
 From the \textcolor{olive}{proposition} and $r\geq 2$, $-K_S = (r-1)H$ is ample. 
\end{proof}
\noindent The following result establishes the boundedness of the index of Fano $3$-folds. 
\begin{proposition}
    Let $X$ have index $r\geq 2$, and suppose that the linear system $|H|$ contains a smooth surface $S$. Then 
    \begin{enumerate}
        \item[(i)] $r \leq 4$;
        \item[(ii)] if $r=2$ then $1 \leq S^3 \leq 9$;
        \item[(ii)] if $r=3$ then $S^3=2$;
        \item[(iv)] if $r=4$ then $S^3=1$.
    \end{enumerate}
\end{proposition}
\begin{proof}
	Let $S \in |H|$ be a del Pezzo surface. Then by  \textcolor{cyan}{cite delpezzo?} 
   \[ 1\leq K_S^2 \leq 9.\]
Plugging in the formula for $K_S$ from 2.3, we get 
\[
1 \leq (r-1)^2 S^3 \geq 9.
\]   
Now $S^3$ is a positive integer	as $-K_X$ is ample and $r\geq 2$, so considering possibilities for positive integer values of $S^3$ gives $r\leq 4$. This proves $(i)$. For $(ii)$ and $(iv)$, using $r=2$ and $4$ respectively gives us possible values of $S^3$. If $r=3$, then from the last inequality $S^3 =1$ or $2$. For $S^3=1$, we get a contradiction
\[
2g-2 = -K_X^3 = (3H)^3 = 27,
\]
so $S^3=2$.		
		
		
\end{proof}

\begin{definition}
    Set $d = d(X) =S^3$. If $H$ is very ample, then $d(X)$ is the \textit{degree} of $\phi_{|H|}(X)$ in $\mathbb{P}^{\text{dim}|H|}$.
\end{definition}
\begin{proposition}
	\begin{enumerate}
		\item[(i)] Pic $X \simeq H^2(X,\mathbb{Z})$.
		\item[(ii)] Pic $X$ is torsion-free. 
	\end{enumerate}
\end{proposition}
\begin{proof}
	The exponential exact sequence 
	\[
	0 \rightarrow \mathbb{Z} \rightarrow \mathcal{O}_X  \xrightarrow{exp} \mathcal{O}_X^* \rightarrow 1
	\]
	 induces the long exact cohomological sequence 
	 \[
	 \dots \rightarrow H^1(X,\mathcal{O}_X) \rightarrow H^1(X,\mathcal{O}_X^*) \xrightarrow{\delta} H^2(X,\mathbb{Z}) \rightarrow H^2(X,\mathcal{O}_X)\rightarrow \dots 
	 \]
	 By Kodaira vanishing theorem, $H^1(X,\mathcal{O}_X) = H^2(X,\mathcal{O}_X) =0$, so $\delta$ is an isomorphism. This shows $(i)$. Let $S \in |-K_X|$. It is a $K3$ surface by \cite{Isk77}, 1.5 and \textcolor{olive}{if?} $r \geq 2$, a del Pezzo surface by 2.4. In any case, $H^2(S,\mathbb{Z})$ is torsion-free. Further by Lefschetz hyperplane theorem, $H^2(X,\mathbb{Z}) \hookrightarrow H^2(S,\mathbb{Z})$. Thus by $(i)$, Pic $X$ is also torsion-free.
\end{proof}
 
\noindent Given a Fano $3$-fold $X$, we label it by a pair of numbers $\rho-N$ where $\rho$ is the Picard rank of $X$ and $N$ is the number in the classification found in \cite{FanoV}. A most recent classification table for Fano $3$-folds along with some of their associated invariants and information about their birational geometry, zero section description(due to \textcolor{cyan}{Fatighenti}) class can be found on \cite{Fano}.

	  
\section{Fano Threefolds with $\rho =1$}
\noindent Let $X$ be a smooth Fano $3$-fold of index $r$. 
\subsection{Fano $3$-folds with $r\geq2$}
The following theorem by Iskovskikh(\cite{Isk77}) completely classifies Fano $3$-folds with index $r\geq 2$. 
\begin{theorem}
    Let $X$ be a Fano $3$-fold of index $r\geq 2$. Then the following assertions hold:
    \begin{enumerate}
        \item[(i)]1-16, 1-17: If $r\geq 3$, then $\phi_{|H|}: X \xrightarrow{\sim} \mathbb{P}^3$ is an isomorphism for $r=4$, and $\phi_{|H|}: X \xrightarrow{\sim} X_2\subset \mathbb{P}^4$ is an isomorphism of $X$ with a smooth quadric of $\mathbb{P}^4$ for $r=3$.
        \item[(ii)] If $r=2$, then a variety $X$ only exists for $1\leq d \leq 7$; for $d\geq 3$, $\phi_{|H|}: X \xrightarrow{\sim}X_d \subset \mathbb{P}^{d+1}$ is an embedding of $X$ as a subvariety $X_d$ of degree $d$ in $\mathbb{P}^{d+1}$, with $X_d$ projectively normal; and if $d\geq 4$, then $X_d$ is the intersection of the quadrics containing it. \\
        Conversely, for any $d\geq 3$, every smooth projectively normal $3$-fold $X_d \subset \mathbb{P}^{d+1}$ not lying in any hyperplane is a Fano $3$-fold, and has index $r=2$, apart from the case $r=4, d=8$, when $X_8$ is the image of $\mathbb{P}^3$ in $\mathbb{P}^9$ under the Veronese embedding.
        \item[(iii)] If $r=2$ and $3\leq d \leq 7$, then; for $d=7$, $X_7$ is the projection of the Veronese $3$-fold $X_8 \subset \mathbb{P}^9$ from some point of $X_8$; \\
        for $d=6$, $X_6 \simeq \mathbb{P}^1 \times \mathbb{P}^1\times \mathbb{P}^1$ in its Segre embedding;\\
        1-15: for $d=5$, $X_5 \subset \mathbb{P}^6$ is unique up to projective equivalence, and can be realized in either of the following two ways:
        \begin{enumerate}
            \item[(a)] as the birational image of a quadric $W \subset \mathbb{P}^4$ under the map defined by the linear system $|\mathcal{O}_W(2)-Y|$ of quadrics passing through a twisted cubic $Y$;
            \item[(b)] as the section of the Grassmannian Gr$(2,5)$ of lines in $\mathbb{P}^4$ by $3$ hyperplanes in general position;
        \end{enumerate}
        1-14: for $d=4$, $X_4$ is any smooth intersection of two quadrics in $\mathbb{P}^5$;\\
        1-13: for $d=3$, $X_3$ is any smooth cubic hypersurface of $\mathbb{P}^4$. 
        \item[(iv)] If $r=2$ and $d=1$ or $2$, then:\\
        1-12: for $d=2$, $\phi_{|H|}: X \to \mathbb{P}^3$ is a double covering with smooth ramification surface $D_4 \subset \mathbb{P}^3$ of degree $4$, and any such variety  is a Fano $3$-fold with $r=2$ and $d=2$; and every Fano $3$-fold with $r=2$ and $d=2$ can be realized as a smooth hypersurface of degree $4$ in $\mathbb{P}(1,1,1,1,2)$;\\
        1-11: for $d=1$, $\phi_{|H|}: X \to \mathbb{P}^2$ is a rational map with a single point of indeterminacy, and with irreducible elliptic fibres; and $X$ can be realized in either of the following two ways:
        \begin{enumerate}
            \item[(a)] $\phi_{|-K_X|}: X \to W_4$ is any double cover of the cone $W_4$ over the Veronese surface $F_4 \subset \mathbb{P}^5$, having smooth ramification divisor $D \subset W_4$ by a cubic hypersurface not passing through the vertex of the cone;
            \item[(b)] any smooth hypersurface of degree $6$ in $\mathbb{P}(1,1,1,2,3)$.
        \end{enumerate}
    \end{enumerate}
\end{theorem}
\noindent Note that in the theorem, for $r=2$, $d=6$, Picard group is Pic $X \simeq \mathbb{Z}^{\oplus 3}$, so in this case $X$ is a Fano $3$-fold with $\rho(X)=3$($3$-$27$). For $r=2, d=7$ is 2-35?\textcolor{red}{Probably not?}

\subsection{Hyperelliptic Fano $3$-folds with $r=1$}
For a Fano $3$-fold of index $r$, Theorem \ref{B} implies that the fundamental linear system $|H|$ is without fixed components and base points and so by \cite{Isk77}, 2.2, it follows that deg $\phi_{|H|} =1$ or $2$. Here we study the case when deg $\phi_{|H|}=2$. 
\begin{definition}
    A Fano $3$-fold $X$ of index $r=1$ is \textit{hyperelliptic} if its anticanonical map $\phi_{|-K_X|}$ is a morphism and is of degree deg $\phi_{|-K_X|} = 2$.
\end{definition}
\begin{theorem}
    Let $X$ be a hyperelliptic Fano variety, and let $\phi_{|-K_X|} : X \to Y \subset \mathbb{P}^{g+1}$ be the corresponding morphism of degree $2$. Then $Y$ is nonsingular and $X$ is uniquely determined by the pair $(Y,D)$, where $D\subset Y$ is the ramification divisor of $\phi_{|-K_X|}$. For $\rho(X)=1$, 
    the pair $(Y,D)$ belongs to one of the following families(and if $D$ is a smooth divisor, then for each pair $(Y,D)$ there exists a Fano $3$-fold $X$):
    \begin{enumerate}
        \item[(i)] 1-1: $Y \simeq \mathbb{P}^3$, and $D$ is a smooth hypersurface of degree $6$; in this case $X$ can be realized alternatively as a smooth hypersurface of degree $6$ in the weighted projective space $\mathbb{P}(1,1,1,1,3)$. 
        \item[(ii)] 1-2 b): $Y \simeq X_2$ is a smooth quadric in $\mathbb{P}^4$ and $D \in |\mathcal{O}_{X_2}(4)|$; that is, $D = X_2 \cap X_4$, where $X_4$ is a quartic in $\mathbb{P}^4$. In this case $X$ can also be realized as a smooth complete intersection in the weighted projective space $\mathbb{P}(1,1,1,1,1,2)$: $X$ is the intersection of a quadric cone and a hypersurface of degree $4$:
        \begin{gather*}
        F_2(x_0,  \dots , x_4)=0 \\
        F_4(x_0, \dots , x_5)=0.
        \end{gather*}
    \end{enumerate}
\end{theorem}
For the proof, see \cite{Isk77}, 7.3-7.6.

\subsection{Fano $3$-folds with $r=1$}	
Let $X$ be a Fano $3$-fold with index $r=1$.
\begin{definition}
    A smooth complete irreducible $3$-fold $X$ over $\mathbb{C}$ is called a \textit{Fano $3$-fold of the principal series} if the anticanonical divisor $-K_X$ is very ample.
\end{definition}
From \textcolor{cyan}{Isk77,78- cite the exact result or conclude}, it follows that all Fano $3$-folds are of the principal series with the exceptions of hyperelliptic Fano $3$-folds, 1-11, and \textcolor{red}{cannot identify Fano $3$-fold in 3.1-b). Also how are these not very ample??} From this point, we will consider our Fano $3$-folds to be of principal series and write $X_{2g-2} \subset \mathbb{P}^{g+1}$ for a Fano $3$-fold of the principal series in its anticanonical embedding.\\
\v{S}okurov showed the existence of line on such Fano $3$-folds with Picard rank $1$ in \cite{Sokuline}. Under this assumption and using the method of double projection from a line, Iskovskikh showed that there exist Fano $3$-folds with $\rho =1$ and $r=1$ for genus $g \leq 10$ and $g=12$ but not for $g=11$.\\
For genus $g=3, 4,$ and $5$, Fano $3$-folds with $r=1$, we have
\begin{proposition}
    A Fano $3$-fold $X_{2g-2}\subset \mathbb{P}^{g+1}$ is a complete intersection only for $g=3, 4$ or $5$, and we have that 
    \begin{enumerate}
    \item[1-2 a):] $X_4 \subset \mathbb{P}^4$ is a quartic hypersurface,
    \item[1-3 :] $X_6 = V_{2\cdot 3}$ is an intersection of a quadric and a cubic in $\mathbb{P}^5$,
    \item[1-4 :] $X_8 = V_{2\cdot 2\cdot 2}$ is an intersection of $3$ quadrics in $\mathbb{P}^6$.
    \end{enumerate}
    Conversely, each smooth complete intersection of the types indicated is a Fano $3$-fold the principal series. 
\end{proposition}
See \cite{Isk78}, 1.3 for a proof.
\textcolor{olive}{To do some computations for genus g, index r and $\rho$ and that will complete $r=\rho=1$ discussion for these genera, I believe. Necessary?}
 
\begin{example}
        ($g=6$) Let us denote by $V$ a section of Pl\"{u}cker embedding Gr$(2,5) \subset \mathbb{P}^9$ by two hyperplanes in general position and a quadric. By adjunction formula, this is a Fano variety of index $1$. \textcolor{olive}{genus, Picard rank computations important? I think I should maybe delete this example here because later in 3.4, Gushel's theorem gives a description of it anyway. my mainreason to include it here to show that Iskovskikh gave genus g=6 Fano $3$-fold too.}.
\end{example}
Iskovskikh asserts that it is possible to show that every Fano $3$-fold $X_{10} \subset \mathbb{P}^7$ with Picard rank $1$ is a section of Grassmannian Gr$(2,5)$. See \cite{Isk78}, 1.5.\\

The following result is the main classification theorem by Iskovskikh for Fano $3$-folds with genus $g\geq 7$. 
\begin{theorem}
    Let $X = X_{2g-2} \subset \mathbb{P}^{g+1}$ be a Fano $3$-fold with $\rho =1$ and $g\geq 7$. %Suppose that $X$ contains a line
    Let $\pi_{2Z}: X \to W \subset \mathbb{P}^{g-6}$ be the double projection from a sufficiently general line $Z \subset X$. Let $E$ denote the hyperplane section of $W$. Then the following assertions hold:
    \begin{enumerate}
        \item[(i)] $g \leq 12$. 
        \item[(ii)] If $g=12$, then $W=W_5 \subset \mathbb{P}^6$ is a prime Fano $3$-fold with index $2$ and degree $5$ (with possibly one singular point); the map $\rho_Y :W \to X$ inverse to $\pi_{2Z}$ is given by the linear system $|3E-2Y|$, with $Y \subset W$ a normal rational curve of degree $5$ in $\mathbb{P}^5$.
        \item[(iii)] There do not exist any prime Fano $3$-folds with $g=11$. 
        \item[(iv)] If $g=10$, then $W= W_2 \subset \mathbb{P}^4$ is a quadric and $\rho_Y : W \to X$ is given by the linear system $|5E-2Y|$, where $Y$ is a smooth curve of genus $2$ and degree $7$ in $\mathbb{P}^4$.
        \item[(v)] If $g=9$, then $W= \mathbb{P}^3$ and $\rho_Y : \mathbb{P}^3 \to X$ is given by the linear system $|7E-2Y|$, where $Y$ is a smooth curve of genus $3$ and degree $7$.
        \item[(vi)] If $g=8$, then $\pi_{2Z}: X \to \mathbb{P}^2$ is a rational map with fibres(after resolving the determinacy) curves of genus $2$, and such that the inverse images of lines of $\mathbb{P}^2$ are rational surfaces.
        \item[(vii)] If $g=7$, then $\pi_{2Z}: X \to \mathbb{P}^1$ is a rational map whose general fiber(after resolving the indeterminacy) is a del Pezzo surface of degree $5$ with $8$ points blown up; $X$ is a rational $3$-fold, and the projection from a line maps it into a complete intersection of $3$ quadrics of $\mathbb{P}^6$ containing a smooth rational ruled surface $R_3 \subset \mathbb{P}^4$.
        \end{enumerate}
\end{theorem}
The proof of $(i)$ and the construction of Fano $3$-folds with given genus in $(ii)$-$(vii)$ uses the birational technique of projection and double projection from lines on $X$. See \cite{Isk78}, $\S 6$ for details.\\
For an alternative proof of boundedness of genus for Fano $3$-folds using vector bundles, see \cite{Muk92}.
	    
\subsection{Mukai and Gushel's description of Fano $3$-folds with $g\geq 6$}
While Iskovskikh's method gives the existence of prime Fano $3$-folds with genus $g \geq 6$, Mukai gave a more explicit description of such projective varieties of dimension $n\geq 3$. 
To classify Fano $n$-folds of the principal series of genus $g \geq 6$, Mukai first showed the existence of a good vector bundle $\mathcal{E}$ on $X$. Then using the linear system $|\mathcal{E}|$, we can embed $X$ into a Grassmannian variety and describe its image. We find that $X$ is a linear section of some homogeneous space. We describe these homogeneous spaces in the following examples.

\begin{example}
	\textcolor{olive}{For Fano $n$-folds, very ample $-K_X$ with $\rho=1$ and genus $g\geq 6$, the dimension $n$ cannot be arbitraily large. In fact, the maximum dimension in these cases $n(g)=24-2g$ is attained by a variety $\Sigma_{2g-2}^{n(g)}$ as below:}
\begin{enumerate}
\item[(i)] $g=7$. Let $V$ be a $9$-dimensional vector space with $F$ as a non degenerate symmetric bilinear form on $V$ and $S$ be the space of spinors of $F$. Here $n(g)=10$. Denote by $\Sigma_{12}^{10} \subset $Gr$(4,9)$ the set of all $4$-dimensional subspaces $W$ of $V$ with $F(W,W)=0$. Then $\Sigma_{12}^{10}$ is a smooth $10$-dimensional subvariety of Grassmannian Gr$(4,9)$ and can be embedded in $\mathbb{P}^{15}$ by the spinor coordinates. \\
Here $\Sigma^{10}_{12} = SO_{10}(\mathbb{C})/P$ is a homogeneous space with $P$ a maximal parabolic subgroup of $SO_{10}(\mathbb{C})$ and is unique up to isomorphism.
\item[(ii)] $g=8$. The Grassmannian $\Sigma^8_{14}:=$ Gr$(2,6) \subset \mathbb{P}(\wedge^2 \mathbb{C}^6) = \mathbb{P}^{14}$ is a smooth Fano variety of dimension $8$ and index $8$.  Here $\Sigma^8_{14} = SL_6(\mathbb{C})/P$ is a homogeneous space with $P$ a maximal parabolic subgroup of $SL_6(\mathbb{C})$.
      \item[(iii)] $g=9$. Let $V$ be a $6$-dimensional vector space and $F$ be a non degenrate skew-symmetric bilinear form on $V$. Here $n(g)=6$. Let us denote by $\Sigma_{16}^6$ the set of all $3$-dimensional subspaces $W$ of $V$ such that $F(W,W)=0$. Then $\Sigma_{16}^6 \subset $ Gr$(3,6)$ is a homogeneous space isomorphic to $U(3)/O(3)$ as varieties, hence is a smooth $6$-dimensional subvariety of degree $16$ in Gr$(3,6)\subset  \mathbb{P}^{19}$.
            \item[(iv)] $g=10$. Let $V$ be a $7$-dimensional vector space and $F$ be a non degenerate skew-symmetric $4$-linear form on $V$. Here $n(g)=5$. Denote by $\Sigma_{18}^5$ the set of all $5$-dimensional subspaces $W$ of $V$ such that $F(W,W,W,W)=0$. Then $\Sigma_{18}^5 \subset $ Gr$(5,7)$ is a homogeneous space, hence a smooth $5$-dimensional subvariety of degree $18$ in Gr$(5,7) \subset \mathbb{P}^{20}$. Here $\Sigma^5_{18}$ is a homogeneous space under the action of the exceptional Lie group of type $G_2$ and is isomorphic to $G_2/P$ where $P$ is a maximal parabolic subgroup of $G_2$.
            
            \item[(v)] $g=12$. Let $V$ be a $7$-dimensional vector space and $F_1$, $F_2$, and $F_3$ be three linearly independent skew-symmetric bilinear forms on $V$. Denote by $X$ the set of $3$-dimensional subspaces $W$ of $V$ with $F_1(W,W)=F_2(W,W)=F_3(W,W)=0$. If the subspace $F_1 \wedge V^{\vee} + F_2 \wedge V^{\vee} + F_3 \wedge V^{\vee}$ of $\wedge^3 V^{\vee}$ contains no vectors of the form $f_1 \wedge f_2 \wedge f_3 \neq 0$ for $f_1, f_2, f_3 \in V^{\vee}$ then $X$ is a smooth $3$-dimensional subvariety of degree $22$, denoted by $\Sigma_{22}^3$. Here $n(g)= 3$.
        \end{enumerate}
\end{example}
\textcolor{olive}{Index computation using mukai92b(uses a result based on root systems), degree of each $\Sigma, 6 \leq g \leq 10$ is $2g-2$, also follows from the same result. As a result we get genus of Fano $X$ is $g$. Picard rank follows from the Lefschetz hyperplane theorem.}
\begin{theorem}
    Let $X$ be a prime Fano $n$-fold ($n \geq 3$) of index $n-2$ and genus $g$, $6\leq g \leq 10$ over $k \subset \mathbb{C}$. Then there exists a $k$-vector space $V$ and a space $M$ of multilinear forms on $V$ such that $X$ is isomorphic to a linear section of $\Sigma_g(V,M) \subset \mathbb{P}^{g+n(g)-2}_k$. 
    For $g=12$, \textcolor{olive}{AG V statement from page 112 or muk89 statement as above? Cannot find that n can only be $3$ in original papers as far as I have understood and read them.}
\end{theorem}
In particular, for $n=3$ and $7 \leq g \leq 10$, a Fano $3$-fold $X_{2g-2} \subset \mathbb{P}^{g+1}$ is obtained as a complete intersection of the homogeneous space $\Sigma_{2g-2}^{n(g)}$ and a linear subspace of codimension $n(g)-3$ in $\mathbb{P}(V) = \mathbb{P}^{g+n(g)-2}$. And by Lefschetz theorem, it follows that such $X=X_{2g-2}$ has Pic $(X) \simeq \mathbb{Z}(-K_X)$.
For Fano $3$-folds of genus $6$ and $8$ over $\mathbb{C}$, the theorem was proved independently by Gushel. For $g=6$, we have  
\begin{theorem}
    Let $X=X_{10} \subset \mathbb{P}^7$ be an anticanonically embedded Fano threefold of index $1$ and genus $6$ with $\rho(X)=1$. Then $X$ is one of the following:
    \begin{enumerate}
        \item[(i)] a section of the Grassmannian $Gr(2,5)$ embedded by Pl\"{u}cker embedding into $\mathbb{P}(\wedge^2 \mathbb{C}^5)$ by a subspace of codimension $2$ and a quadric,
        \item[(ii)] the section by a quadric of a cone $W=W_5 \subset\mathbb{P}^7$ over a nonsingular del Pezzo threefold $V= V_5 \subset \mathbb{P}^6$ of degree $5$.
    \end{enumerate}
    Threefolds of type $(i)$ and $(ii)$ are not isomorphic.
\end{theorem}
See \cite{Gus6} for details on the proof.
A similar method can be applied to study Fano threefolds of genus $8$, see \cite{Gus83} and \cite{Gus92}. 
\begin{theorem}
    Let $X=X_{14} \subset \mathbb{P}^9$ be an anticanonically embedded Fano threefold of index $1$ and genus $8$ with $\rho(X)=1$. Then $X$ is a section of the Grassmannian Gr$(2,6)$ embedded by Pl\"{u}cker embedding into $\mathbb{P}(\wedge^2 \mathbb{C}^6) = \mathbb{P}^{14}$ by a subspace of codimension $5$.
\end{theorem}

As a consequence, we see that the Fano $3$-folds with Picard rank $\rho =1$ appear as one of the following: 
\begin{enumerate}
    \item[(i)] Sections of Grassmannians(1-5 a), 1-6, 1-7, 1-8, 1-9, 1-10, 1-15)
    \item[(ii)] Complete intersections in projective or weighted projective space (1-2 a), 1-3, 1-4, 1-11, 1-12, 1-13, 1-14, 1-16, \textcolor{olive}{1-17, 1-11, 1-12})
    \item[(iii)] Hyperelliptic (1-1, 1-2 b))
\end{enumerate}
\textcolor{red}{to decide about 1-5b) after Gushel}
	
	
	
	
	
	
\section{Fano Threefolds with $\rho \geq 2$}
	\textcolor{olive}{outline from mm(1981). Some details from others yet to add because what I have now is just not sufficient.}\\ 
	In this section, we consider Fano $3$-folds with $B_2 \geq 2$. The main result is 
	\begin{theorem}
	There are exactly $88$ types of Fano $3$-folds with $B_2 \geq 2$ up to deformations.
	\end{theorem}
	We begin with some definitions. 
	\begin{definition}
		A Fano $3$-fold is imprimitive if it is isomorphic to the blow-up of a Fano $3$-fold along a smooth irreducible curve. A Fano $3$-fold is primitive if it is not imprimitive.
	\end{definition}
    \begin{definition}
    	A smooth variety over a smooth surface $S$ is a conic bundle if every geometic fibre of $X \to S$ is isomorphic to a conic, i.e., a scheme of zeroes of a non zero homogeneous form of degree $2$ on $\mathbb{P}^2$.
    \end{definition}
    \noindent The following theorem gives a complete classification for primitive Fano $3$-folds.
    \begin{theorem}
    	Let $X$ be a primitive Fano $3$-fold. Then we have 
    	\begin{enumerate}
    		\item $B_2 \leq 3$,
		\item if $B_2 =2$, then $X$ is a conic bundle over $\mathbb{P}^2$, and
    		\item if $B_2 =3$, then $X$ is a conic bundle over $\mathbb{P}^1 \times \mathbb{P}^1$ and has either a divisor $D \simeq \mathbb{P}^1 \times \mathbb{P}^1$ such that $\mathcal{O}_D(D) \simeq \mathcal{O}(-1,-1)$ or another conic bundle structure over $\mathbb{P}^1 \times \mathbb{P}^1$.
    	\end{enumerate}
    \end{theorem}
      \noindent The following proposition is important for classifying imprimitive Fano $3$-folds.
    \begin{proposition}
    	On a Fano $3$-fold $X$ with $B_2 =2$, there are two smooth rational curves $C_1$ and $C_2$ and two numerically effective divisors $H_1$ and $H_2$ such that $(C_i \cdot H_j)= \delta_{ij}$ for all $i,j=1,2$. 
    \end{proposition}
    \noindent  
    It turns out that imprimitive Fano $3$-folds can be obtained from successive curve-blow-ups of primitive Fano $3$-folds by using their conic bundle structure or the existence of lines on Fano $3$-folds with $B_2=2$. In the latter case, there can be several possibilities for each of the extremal rays and each possibility leads to an imprimitive Fano $3$-fold with $B_2=2$. Combining all this information we get Table 2 in \textcolor{cyan}{refer tables}.\\
    \noindent Since the blowing-up of a Fano $3$-fold raises $B_2$ by $1$(refer?), we obtain $B_2 \geq 3$ imprimitive Fano $3$-folds by the blowing-up of a Fano $3$-fold $Y$ along a smooth irreducible curve $C$. The following Propositions give strong necessary conditions on $C \subset Y$.
    

\begin{thebibliography}{12}
\bibitem{Isk77}
V.A. Iskovskih, 
\newblock {\em Fano 3-folds I},
\newblock Izv. Akad. Nauk SSSR Ser. Mat.  \textbf{41} (1977), 516-562; English transl. in Math. USSR Izv. \textbf{11} No. 3 (1977).

\bibitem{Isk78}
V. A. Iskovskih, 
\newblock {\em Fano 3-folds II}, 
\newblock Izv. Akad. Nauk SSSR Ser. Mat.
\textbf{42} (1978), 506-549; English transl. in Math. USSR Izv. \textbf{12} No. 3 (1978), 469-506.

\bibitem{Sokudivisor}
V. V. \v{S}okurov, 
\newblock{\em The smoothness of the general anticanonical divisor on a Fano variety},
\newblock Izv. Akad. Nauk SSSR Ser. Mat. \textbf{43} (1979), 430-441; English transl. in Math. USSR Izv. \textbf{14} No. 2 (1980).

\bibitem{Sokuline}
V. V, \v{S}okurov, 
\newblock {\em The Existence of lines on Fano threefolds},
\newblock English transl. in Math. USSR Izv. \textbf{15} (1980), 173-209.

\bibitem{Hartshorne}
R. Hartshorne.
\newblock {\em Algebraic Geometry},
\newblock Graduate Texts in Mathematics, No. 52, Springer-Verlag, New York-Heidelberg, 1977.
	
\bibitem{FanoV}
V. A. Iskovskih, Yu. G. Prokhorov, Fano varieties,
\newblock{ \em Algebraic Geometry V},
\newblock Encyclopaedia Math. Sci., vol. 47, Springer, Berlin 1999.

\bibitem{Muk89}
S. Mukai,
\newblock{\em Biregular classification of Fano 3-folds and Fano manifolds of coindex 3},
\newblock Proc. Nat. Acad. Sci. U.S.A., \textbf{86}, No. 9 (1989), 3000-3002.

\bibitem{Muk92}
S. Mukai, (1992). 
\newblock{\em Fano 3-folds},
\newblock In G. Ellingsrud, C. Peskine, G. Sacchiero, \& S. Stromme (Eds.), Complex Projective Geometry: Selected Papers (London Mathematical Society Lecture Note Series, pp. 255-263). Cambridge: Cambridge University Press. 

\bibitem{Fano}
Pieter Belmans,
\newblock{\em https://www.fanography.info/}

\bibitem{Gus6}
N. P. Gushel,
\newblock{\em Fano varieties of genus $6$},
\newblock Izv. Akad. Nauk. SSSR Ser. Mat. \textbf{46} (1982), 1159-1174 (Russian); English transl. in Math. USSR  Izv. \textbf{21} (1983), 445-459.

\bibitem{Gus83}
N. P. Gushel,
\newblock{\em Fano varieties of genus $8$},
\newblock Usp. Mat. Nauk \textbf{38} (1983), 163-164 (Russian); English transl. in Russ. Math. Surv. \textbf{38} (1983), 192-193. 

\bibitem{Gus92}
N. P. Gushel,
\newblock{\em Fano $3$-folds of genus $8$},
\newblock Algebra i Analiz \textbf{4} (1992), 120-134 (Russian); English transl. in St. Petersbg. Math. J. \textbf{4} (1993), 115-129. 


\end{thebibliography}
\end{document}
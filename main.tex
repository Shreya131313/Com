\documentclass[12pt]{amsart}
\usepackage[utf8]{inputenc}
\usepackage{amssymb,amsmath,amsthm}
\usepackage[margin=1in]{geometry}
\usepackage{xcolor}


\theoremstyle{plain}
\newtheorem{theorem}{Theorem}[section]
\newtheorem{corollary}[theorem]{Corollary}
\newtheorem{lemma}[theorem]{Lemma}
\newtheorem{fact}[theorem]{Fact}
\newtheorem{proposition}[theorem]{Proposition}

\theoremstyle{definition}
\newtheorem{definition}[theorem]{Definition}
\newtheorem{remark}[theorem]{Remark}
\newtheorem{question}[theorem]{Question}
\newtheorem{conjecture}[theorem]{Conjecture}
\newtheorem{convention}[theorem]{Convention}
\newtheorem{example}[theorem]{Example}


\newtheoremstyle{expl}%
{0ex}% Space above
{0ex}% Space below
{}% Body font
{}% Indent amount
{\bfseries}% Theorem head font
{.}% Punctuation after theorem head
{.5em}% Space after theorem head
{}% Theorem head spec (can be left empty, meaning `normal')
\theoremstyle{expl}

\begin{document}
	\title{\large{Comprehensive Exam}}
	\date{}
	\author{\small{Shreya Sharma}}
	\maketitle
	\tableofcontents
	
\section{Personal Background}
\noindent Basically Copy.
\section{Introduction}
\textcolor{olive}{Notations, definition(index, $\rho$, degree, etc., extremal rays and stuff?), basic examples, goal(s), other preliminaries about invariants, definitions, etc., potential research?}
\subsection{Definition and Examples.} Throughout we work over the field of complex numbers $\mathbb{C}$.
A nonsingular projective variety $X$ is called a \textit{Fano Variety} if the anticanonical divisor $-K_X$ is ample.
For any Cartier divisor $D$ on a variety $X$, $\mathcal{O}_X(D)$ will denote the corresponding invertible sheaf, and, in particular, $\mathcal{O}_X(-K_X)$ is the canonical sheaf on $X$ where $-K_X$ is the canonical divisor of $X$. \\
	\textcolor{red}{ Q.2 Complete linear system, $h^i(X)$, Proposition 1.3, build upto Prop 1.6 \& as a consequence genus, and the following para- NS(X) =Pic(X) has no torsion, $\rho=b_2$, fundamental divisor $H$, index. Then Def 1.13 for degree.}\\
For any positive integer $n$,  $\mathbb{P}^n$ is a $n$-dimensional Fano variety since its anticanonical sheaf is $\mathcal{O}(n+1)$ which is ample by \textcolor{cyan}{Hart}. In fact $\mathbb{P}^1$ is the only 1-dimensional Fano variety. Fano varities of dimension $2$ are called \textit{del Pezzo surfaces} and their classification is given in \textcolor{cyan}{AG V}. Clearly, $\mathbb{P}^3$ is a $3$-dimensional Fano variety. + 1 more example of hypersurfaces.

\noindent Fano varieites of dim $3$ with $\rho =1$ are called \textit{prime} Fano $3$-folds and they are completely classified by Iskovskikh in \textcolor{cyan}{Isko-1,2}. Fano $3$-folds with $\rho\geq 2$ were all classified by mori and mukai in \textcolor{cyan}{refer with timeline?}. \\
In this paper(?), we aim to understand the classification of Fano $3$-folds. 

	  
\section{Fano Threefolds with $\rho =1$}
	\textcolor{olive}{ $V_d$ notation, etc that will be used in next section}\\
	For the classification here, we follow \textcolor{cyan}{Isko-1,2}.
\section{Fano Threefolds with $\rho \geq 2$}
	\textcolor{olive}{outline from mm(1981). Some details from others yet to add because what I have now is just not sufficient.}\\ 
	In this section, we consider Fano $3$-folds with $B_2 \geq 3$. The main result is 
	\begin{theorem}
	There are exactly $88$ types of Fano $3$-folds with $B_2 \geq 2$ up to deformations.
	\end{theorem}
	We begin with some definitions. 
	\begin{definition}
		A Fano $3$-fold is imprimitive if it is isomorphic to the blow-up of a Fano $3$-fold along a smooth irreducible curve. A Fano $3$-fold is primitive if it is not imprimitive.
	\end{definition}
    \begin{definition}
    	A smooth variety over a smooth surface $S$ is a conic bundle if every geometic fibre of $X \to S$ is isomorphic to a conic, i.e., a scheme of zeroes of a non zero homogeneous form of degree $2$ on $\mathbb{P}^2$.
    \end{definition}
    \noindent The following theorem gives a complete classification for primitive Fano $3$-folds.
    \begin{theorem}
    	Let $X$ be a primitive Fano $3$-fold. Then we have 
    	\begin{enumerate}
    		\item $B_2 \leq 3$,
		\item if $B_2 =2$, then $X$ is a conic bundle over $\mathbb{P}^2$, and
    		\item if $B_2 =3$, then $X$ is a conic bundle over $\mathbb{P}^1 \times \mathbb{P}^1$ and has either a divisor $D \simeq \mathbb{P}^1 \times \mathbb{P}^1$ such that $\mathcal{O}_D(D) \simeq \mathcal{O}(-1,-1)$ or another conic bundle structure over $\mathbb{P}^1 \times \mathbb{P}^1$.
    	\end{enumerate}
    \end{theorem}
      \noindent The following proposition is important for classifying imprimitive Fano $3$-folds.
    \begin{proposition}
    	On a Fano $3$-fold $X$ with $B_2 =2$, there are two smooth rational curves $C_1$ and $C_2$ and two numerically effective divisors $H_1$ and $H_2$ such that $(C_i \cdot H_j)= \delta_{ij}$ for all $i,j=1,2$. 
    \end{proposition}
    \noindent  
    It turns out that imprimitive Fano $3$-folds can be obtained from successive curve-blow-ups of primitive Fano $3$-folds by using their conic bundle structure or the existence of lines on Fano $3$-folds with $B_2=2$. In the latter case, there can be several possibilities for each of the extremal rays and each possibility leads to an imprimitive Fano $3$-fold with $B_2=2$. Combining all this information we get Table 2 in \textcolor{cyan}{refer tables}.\\
    \noindent Since the blowing-up of a Fano $3$-fold raises $B_2$ by $1$(refer?), we obtain $B_2 \geq 3$ imprimitive Fano $3$-folds by the blowing-up of a Fano $3$-fold $Y$ along a smooth irreducible curve $C$. The following Propositions give strong necessary conditions on $C \subset Y$.
    
	\section{References}
\end{document}
